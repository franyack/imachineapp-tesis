%---------------------------------------------------------------------
%
%                          Parte 2
%
%---------------------------------------------------------------------
%
% Parte2.tex
% Copyright 2009 Marco Antonio Gomez-Martin, Pedro Pablo Gomez-Martin
%
% This file belongs to the TeXiS manual, a LaTeX template for writting
% Thesis and other documents. The complete last TeXiS package can
% be obtained from http://gaia.fdi.ucm.es/projects/texis/
%
% Although the TeXiS template itself is distributed under the 
% conditions of the LaTeX Project Public License
% (http://www.latex-project.org/lppl.txt), the manual content
% uses the CC-BY-SA license that stays that you are free:
%
%    - to share & to copy, distribute and transmit the work
%    - to remix and to adapt the work
%
% under the following conditions:
%
%    - Attribution: you must attribute the work in the manner
%      specified by the author or licensor (but not in any way that
%      suggests that they endorse you or your use of the work).
%    - Share Alike: if you alter, transform, or build upon this
%      work, you may distribute the resulting work only under the
%      same, similar or a compatible license.
%
% The complete license is available in
% http://creativecommons.org/licenses/by-sa/3.0/legalcode
%
%---------------------------------------------------------------------

% Parte en donde se explica lo "copado" que es Learninspy

\partTitle{Learninspy}

\partDesc{Esta segunda parte de la tesis est� dedicada a detallar todas las caracter�sticas del framework implementado. Se describe la arquitectura escogida para su implementaci�n, justificando las elecciones de dise�o realizadas, y las propiedades que lo caracterizan como sistema para modelar redes neuronales profundas en forma distribuida. A su vez, se detalla una evaluaci�n realizada sobre el mismo para validar su correcto funcionamiento y argumentar la bondad de sus caracter�sticas respecto a los objetivos planteados.}

\partBackText{}

\makepart
