%---------------------------------------------------------------------
%
%                          Ap�ndice 1
%
%---------------------------------------------------------------------

\chapter{Algoritmos}
\label{ap1:algoritmos}


\begin{algorithm}
	\caption{Algoritmo de retropropagaci�n para redes neuronales}\label{alg:backpropagation}
	\begin{algorithmic}
		\Require Matrices de pesos sin�pticos $W$, vectores de bias $b$, n\textordmasculine de capas $L$.
		\Function{backpropagation}{$x$, $y$} \Comment{\textit{Par�metros}: Patr�n de entrada $x$; salida deseada $y$.}
		\State \textbf{1) Entrada}: 
		\State $a^{(1)} = f(x) $ \Comment{Calcular la activaci�n para la capa de entrada }
		\State \textbf{2) Paso hacia adelante}: 
		\For{$l = 2, 3, \dots, L$}
		\State $z^{(l)} = W^{(l)}a^{(l-1)} + b^{(l)}$ \Comment{Salida lineal de la capa.}
		\State $a^{(l)} = f(z^{(l)})$ \Comment{Salida activada de la capa.}
		\EndFor
		\State \textbf{3) Error en salida} 
		\State \( \phantom{\nabla_a J} \mathllap{J} = loss(a^{(L)}, y)  \) \Comment{Aplicar funci�n de costo}
		\State \(\nabla_a J = d\_loss(a^{(L)}, y)\) \Comment{Gradiente del costo respecto a la activaci�n.}
		\State \(\phantom{\nabla_a J} \mathllap{\delta^L} = \nabla_a J \odot f'(z^{(l)})\)
		\Comment{Computar el vector derivada de la salida}
		\State \textbf{4) Retropropagar el error y computar gradientes}:
		\For{$l = L-1, L-2, \dots, 2$}
		\State \( \phantom{\nabla_{W^{(l)}} J} \mathllap{ \delta^{(l)}}= (W^{(l+1)})^{\top} \delta^{(l+1)} \odot f'(z^{(l)}) \)
		\State \( \nabla_{W^{(l)}} J = \delta^{(l+1)} (a^{(l)})^\top \) \Comment{Respecto al par�metro $W$ de la capa $l$} 
		\State \( \phantom{\nabla_{W^{(l)}} J} \mathllap{\nabla_{b^{(l)}} J} = \delta^{(l+1)} \) \Comment{Respecto al par�metro $b$ de la capa $l$} 
		\EndFor
		\State \Return $J, \nabla_{W} J, \nabla_{b} J  $ \Comment{Devuelve el costo de la red y los gradientes}
		\EndFunction
	\end{algorithmic}
\end{algorithm}


\begin{algorithm}
	\caption{Salida de una capa de neuronas a la que se le aplica Dropout}\label{alg:dropout}
	\begin{algorithmic}
		\Require Capa de neuronas $l_{W,b}$.
		\Function{dropoutput}{$x$, $p$} \Comment{\textit{Par�metros}: patr�n de entrada $x$; probabilidad de activaci�n $p \in (0, 1)$.}
		\State \text{$a = output(l_{W,b}$, $x$) }  \Comment{Computar salida de capa $l_{W,b}$ ante la entrada $x$.}
		\State \text{$v = random(size(a$))} \Comment{Generar vector del mismo tama�o que $a$, con valores aleatorios entre 0 y 1.}
		\State \text{$m = v > p$} \Comment{Calcular m�scara binaria del vector $v$, donde se asigna 1 a los valores de $v$ mayores a $p$ y 0 al resto.}
		\State \text{$d = (a \odot m) / p$} \Comment{Aplicar m�scara binaria sobre el vector de activaciones, escalando a $p$.}
		\State \Return{$d, m$} \Comment{Devuelve la salida de la red con Dropout aplicado, y la m�scara binaria correspondiente.}
		\EndFunction
	\end{algorithmic}
\end{algorithm}



% Variable local para emacs, para  que encuentre el fichero maestro de
% compilaci�n y funcionen mejor algunas teclas r�pidas de AucTeX
%%%
%%% Local Variables:
%%% mode: latex
%%% TeX-master: "../Tesis.tex"
%%% End:
