%---------------------------------------------------------------------
%
%                      resumen.tex
%
%---------------------------------------------------------------------
%
% Contiene el cap�tulo del resumen.
%
% Se crea como un cap�tulo sin numeraci�n.
%
%---------------------------------------------------------------------

\chapter{Resumen}
\cabeceraEspecial{Resumen}

\begin{FraseCelebre}
\begin{Frase}
No basta tener un buen ingenio, lo principal es aplicarlo bien.
\end{Frase}
\begin{Fuente}
Ren� Descartes
\end{Fuente}
\end{FraseCelebre}

En la actualidad, el constante avance de las tecnolog�as de comunicaci�n
permite que la transmisi�n de informaci�n multimedia sea un aspecto com�n y diario
para las personas que utilizan dispositivos m�viles. El fen�meno de las redes sociales,
trae consigo un tr�fico de datos que muchas veces se hace dif�cil de administrar y
controlar por los usuarios, produciendo que se alcance el l�mite de almacenamiento en
los dispositivos, principalmente en aquellos de gama baja o media, aunque
ocasionalmente sucede tambi�n en los �ltimos modelos. Por otro lado, cuando se
quieren administrar los archivos de un dispositivo m�vil, o bien al realizar una copia de
seguridad, las tareas llevadas a cabo para organizar las im�genes consumen mucho
tiempo. 

Gracias a los grandes avances que hoy en d�a provee la visi�n artificial, una
disciplina que combina t�cnicas de procesamiento de im�genes y aprendizaje
maquinal, se considera posible lograr un sistema que cuente con las propiedades
adecuadas para lograr un agrupamiento o \textit{clustering} de datos, bas�ndose en una
similitud computada sobre lo mismos. Con ello, se podr�a sugerir al usuario una
estructura de directorios para organizar las im�genes de su dispositivo, y a partir de ah�
proveer funcionalidades para agilizar y automatizar las tareas de administraci�n
involucradas, reduciendo el tiempo y esfuerzo requerido.

Por lo tanto, la propuesta de este proyecto es implementar una aplicaci�n que
agrupe, de manera autom�tica, las im�genes de un directorio seleccionado por el
usuario. Luego, una vez realizado el procesamiento, �ste podr� tomar la decisi�n que
desee interactuando con el resultado (e.g eliminar un grupo obtenido, fusionarlo con
otros, renombrar carpetas, ingresar en un grupo y eliminar alguna imagen en particular,
etc.). Dicha aplicaci�n ser� implementada para la plataforma Android, dado que es un
sistema operativo muy popular y adem�s gratuito.

\smallskip
\noindent \textbf{Palabras claves:} im�genes, agrupamiento, administraci�n y control, similitud computada.

\endinput
% Variable local para emacs, para  que encuentre el fichero maestro de
% compilaci�n y funcionen mejor algunas teclas r�pidas de AucTeX
%%%
%%% Local Variables:
%%% mode: latex
%%% TeX-master: "../Tesis.tex"
%%% End:
