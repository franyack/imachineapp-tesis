%---------------------------------------------------------------------
%
%                      agradecimientos.tex
%
%---------------------------------------------------------------------
%
% agradecimientos.tex
% Copyright 2009 Marco Antonio Gomez-Martin, Pedro Pablo Gomez-Martin
%
% This file belongs to the TeXiS manual, a LaTeX template for writting
% Thesis and other documents. The complete last TeXiS package can
% be obtained from http://gaia.fdi.ucm.es/projects/texis/
%
% Although the TeXiS template itself is distributed under the 
% conditions of the LaTeX Project Public License
% (http://www.latex-project.org/lppl.txt), the manual content
% uses the CC-BY-SA license that stays that you are free:
%
%    - to share & to copy, distribute and transmit the work
%    - to remix and to adapt the work
%
% under the following conditions:
%
%    - Attribution: you must attribute the work in the manner
%      specified by the author or licensor (but not in any way that
%      suggests that they endorse you or your use of the work).
%    - Share Alike: if you alter, transform, or build upon this
%      work, you may distribute the resulting work only under the
%      same, similar or a compatible license.
%
% The complete license is available in
% http://creativecommons.org/licenses/by-sa/3.0/legalcode
%
%---------------------------------------------------------------------
%
% Contiene la p�gina de agradecimientos.
%
% Se crea como un cap�tulo sin numeraci�n.
%
%---------------------------------------------------------------------

\chapter{Agradecimientos}

\cabeceraEspecial{Agradecimientos}

\begin{FraseCelebre}
	\begin{Frase}
	Sentir gratitud y no expresarla, es como envolver un regalo y no darlo.
	\end{Frase}
	\begin{Fuente}
	William Arthur Ward
	\end{Fuente}
\end{FraseCelebre}

Al momento de hacer un balance sobre las experiencias y cosas que he vivido durante mi carrera universitaria, quedan en mi memoria muchos recuerdos hermosos y positivos, sobre todo, por aquellas personas a las que conoc� y que hoy forman parte de mi vida. 

Quiero agradecer profundamente y en primer lugar a mi pap� Oscar y a mi mam� Rosa, quienes con su confianza y amor creyeron en mi, me brindaron su total y completo apoyo para que yo pueda irme a estudiar a una ciudad diferente a la que vivo, con el objetivo de poder formarme profesionalmente. Sin lugar a dudas que sin ellos, esto hubiera sido imposible. A mi hermana Mar�a, con quien compartimos la experiencia de vivir juntos y acompa�arnos en todo tipo de situaciones, la cual nunca dud� en estar presente por cualquier cosa que necesite. A mi novia Sofia, quien con su amor, contenci�n y presencia supo darme aliento y confianza en aquellos momentos de flaqueza, festejando en las victorias y acompa�ando en las ca�das. A mis amigos, la familia que uno elige, por el cari�o de tantos a�os y la confianza de saber que siempre van a estar presentes.

Me gustar�a adem�s agradecer a cada una de las personas que componen la Facultad de Ingenier�a y Ciencias H�dricas. A todos los profesores que con su conocimiento, experiencias y formas marcaron el camino para poder hoy estar presentando mi proyecto final de carrera. Me llena de orgullo decir que yo estudi� en esta facultad.

Tambi�n mi gratitud va hacia mis directores de proyecto, excelentes profesionales y pilares fundamentales en el desarrollo de esta Tesis, ya que sin ellos hubiera sido pr�cticamente imposible para m� llevarla adelante de la manera en la que lo hicimos.

Por �ltimo y no menos importante, a los amigos que me llevo de esta hermosa experiencia: Agu, Lautaro, Facu S., Foco, Damu, Tinky, Facu C., Tomi, Guido, Gian y Pico. Gracias por siempre brindarme su conocimiento, tener la mejor predisposici�n y luchar codo a codo sabiendo que esta carrera, si no se la hace en equipo, se hace much�simo m�s dif�cil de llegar.

\endinput
% Variable local para emacs, para  que encuentre el fichero maestro de
% compilaci�n y funcionen mejor algunas teclas r�pidas de AucTeX
%%%
%%% Local Variables:
%%% mode: latex
%%% TeX-master: "../Tesis.tex"
%%% End:
