%---------------------------------------------------------------------
%
%                      agradecimientos.tex
%
%---------------------------------------------------------------------
%
% agradecimientos.tex
% Copyright 2009 Marco Antonio Gomez-Martin, Pedro Pablo Gomez-Martin
%
% This file belongs to the TeXiS manual, a LaTeX template for writting
% Thesis and other documents. The complete last TeXiS package can
% be obtained from http://gaia.fdi.ucm.es/projects/texis/
%
% Although the TeXiS template itself is distributed under the 
% conditions of the LaTeX Project Public License
% (http://www.latex-project.org/lppl.txt), the manual content
% uses the CC-BY-SA license that stays that you are free:
%
%    - to share & to copy, distribute and transmit the work
%    - to remix and to adapt the work
%
% under the following conditions:
%
%    - Attribution: you must attribute the work in the manner
%      specified by the author or licensor (but not in any way that
%      suggests that they endorse you or your use of the work).
%    - Share Alike: if you alter, transform, or build upon this
%      work, you may distribute the resulting work only under the
%      same, similar or a compatible license.
%
% The complete license is available in
% http://creativecommons.org/licenses/by-sa/3.0/legalcode
%
%---------------------------------------------------------------------
%
% Contiene la p�gina de agradecimientos.
%
% Se crea como un cap�tulo sin numeraci�n.
%
%---------------------------------------------------------------------

\chapter{Agradecimientos}

\cabeceraEspecial{Agradecimientos}

\begin{FraseCelebre}
	\begin{Frase}
	Dame un punto de apoyo \\y mover� el mundo.
	\end{Frase}
	\begin{Fuente}
	Arqu�medes
	\end{Fuente}
\end{FraseCelebre}


Quisiera agradecer profundamente a cada uno de los actores que fueron parte de mi paso por esta excelente facultad, tanto profesores como colegas estudiantes, por ofrecer su tiempo y conocimientos para formar en m� un futuro profesional, desde lo acad�mico hasta como persona. Principalmente quiero destacar el trabajo del instituto \textit{sinc(i)}, por la ayuda que cada uno de sus integrantes me brind� por varios a�os, fundamentalmente de mis directores quienes me iniciaron en este camino que hoy compone mi vocaci�n laboral como \textit{data scientist}.  

Tambi�n quiero dar las gracias a todos mis amigos que supieron estar conmigo durante esta etapa de mi vida en todos sus aspectos. Algunos de toda mi vida, otros de muchos a�os, y algunos otros que adquir� durante mis estudios y que acompa�aron mi trayecto en la universidad. Todos ellos son una gran riqueza que afortunadamente conserv� durante estos a�os y pretendo preservar a trav�s de las siguientes etapas de mi vida.

Pero todo este apoyo que recib� en estos a�os no se compara con el brindado por mi hermosa familia. Agradezco de coraz�n a mi hermana Joana que en cada momento me transmiti� con su ejemplo responsabilidad y dedicaci�n para formar mi vocaci�n; a mi padre Javier que supo inspirar en m� gran parte de la cultura que hoy asumo en mi personalidad; y a mi madre Cristina que con mucha ternura me ayud� a progresar y siempre me acompa�a a perseguir mis sue�os. Todos ellos supieron darme calidez y amor de familia que alimentaron mi crecimiento, y a ellos les debo cada uno de los pasos que me llevaron a la persona que soy. 

Finalmente debo agradecer a Dios, quien siempre sabe sembrar bendiciones en mi camino para poder cumplir todas mis metas, y gracias a quien tengo el privilegio de concluir esta linda etapa de mi vida y acompa�ado del excelente entorno de personas que mencion� anteriormente.

\endinput
% Variable local para emacs, para  que encuentre el fichero maestro de
% compilaci�n y funcionen mejor algunas teclas r�pidas de AucTeX
%%%
%%% Local Variables:
%%% mode: latex
%%% TeX-master: "../Tesis.tex"
%%% End:
